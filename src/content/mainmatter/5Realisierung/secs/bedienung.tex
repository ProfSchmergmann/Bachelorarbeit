\section{Bedienung}\label{sec:bedienung}

	Falls ein Client beim \nameref{subsec:autorisierungsserver} registriert ist
	und Rechte, Bereiche sowie eine ID zugewiesen bekommen hat,
	so funktioniert die Bedienung immer genauso wie es in \vref{fig:appsequence} dargestellt ist.
	Zuerst muss der Client ein \gls{jwt} beim \nameref{subsec:autorisierungsserver} beantragen,
	welcher dieses nach erfolgreicher Validierung als Antwort zurückschickt.
	Danach folgt erst die eigentliche Anfrage der Daten.
	Die Abbildungen \nameref{fig:http-methoden-S1},
	\nameref{fig:http-methoden-S2}
	und \nameref{fig:http-methoden-S3} zeigen die momentan
	verfügbaren \glspl{url} des \hyperref[subsec:webservice]{Webservices},
	welche alle
	-- bis auf \inlinelst{/api/Coun\-tries/full\-Anony\-mous}
	\footnote{Die Route \inlinelst{/api/Countries/fullAnonymous} wurde zu Testzwecken eingeführt,
		um vor allem die \vref{fig:timeOfRequests} erstellen zu können.} --
	nur über eine vorherige \gls{autorisierung} erreichbar sind.
	Der Client muss nun also zusammen mit dem \gls{jwt} im Header eine Anfrage an eine dieser Methoden losschicken,
	welche dann vom \nameref{subsec:webservice} weiter bearbeitet werden kann.
	Dieser lässt zuerst das Token durch den \nameref{subsec:autorisierungsserver} validieren,
	um bei Erfolg die Daten bei der Datenbank des \lfidSystems{} über eine \gls{sql}-Anfrage abfragen zu können.
	Falls die Daten erfolgreich aus der Datenbank gelesen wurden,
	werden diese wieder als \gls{json}-Objekt an den Clienten zurückgeschickt.
