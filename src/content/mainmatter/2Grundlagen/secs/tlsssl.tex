\section{TLS/SSL}\label{sec:tls/ssl}

	Um die Übertragung der Daten beispielsweise durch das \gls{http} sicher zu gestalten
	und vor dem Zugriff Dritter zu schützen,
	wird das \gls{tls}-Protokoll genutzt.
	Der noch meistens äquivalent verwendete Name \gls{ssl} wurde 1999 offiziell
	durch die Veröffentlichung von \enquote{The TLS Protocol Version 1.0}\footcite[Vgl.][]{rfc2246}
	durch die \gls{ietf} in \gls{tls} umbenannt.
	Diese Version gilt allerdings mittlerweile als ungültig und sollte daher nicht mehr genutzt werden.
	Aktuell werden die Versionen 1.2\footcite[Vgl.][]{rfc5246} und 1.3\footcite[Vgl.][]{rfc8446} genutzt.
	Im Kern werden für die Absicherung zwei interne Protokolle genutzt,
	welche im Folgenden erläutert werden.

	\subsection{TLS Handshake Protocol}\label{subsec:tls-handshake-protocol}
	Alle \gls{tls} Handshakes nutzen ein asymmetrisches Verschlüsselungsverfahren.
	Das Protokoll durchläuft dafür folgende Schritte:

	\begin{compactenum}
		\item Der Client schickt eine \enquote{Hello} Nachricht an den Server
		und leitet somit den Handshake ein.
		Die Nachricht besteht aus der unterstützten \gls{tls}-Version,
		den unterstützten \glspl{cipher-suite}
		und einer zufälligen Anordnung von Bytes,
		welche als \enquote{Client Random} bezeichnet werden.

		\item Die Antwort vom Server,
		auch als \enquote{Server-Hello} bezeichnet,
		enthält das \gls{ssl-zertifikat},
		die ausgewählte \gls{cipher-suite} und ein \enquote{Server Random},
		welches ebenfalls eine Byte-Folge ist.

		\item Der Client prüft daraufhin das \gls{ssl-zertifikat} bei der Zertifizierungsstelle.

		\item Anschließend wird das \enquote{Premaster Secret},
		eine zufällig generierte Byte-Folge,
		mit dem im \gls{ssl-zertifikat} enthaltenen öffentlichen Schlüssel verschlüsselt
		und an den Server geschickt,
		welcher dieses nur mit dem privaten Schlüssel wieder entschlüsseln kann.

		\item Der Server entschlüsselt nun das \enquote{Premaster Secret},
		woraufhin beide Partner einen Sitzungsschlüssel aus \enquote{Client Random},
		\enquote{Server Random} und \enquote{Premaster Secret} generieren.

		\item Nun senden Client und Server jeweils eine \enquote{Fertig}-Nachricht,
		welche beide mit dem Sitzungsschlüssel verschlüsselt sind.

		\item Der Handshake ist abgeschlossen
		und die weitere Kommunikation wird mit dem Sitzungsschlüssel fortgesetzt.

	\end{compactenum}
	\subsection{TLS Record Protocol}\label{subsec:tls-record-protocol}
	Das Record Protokoll wird für die eigentliche Datenübertragung genutzt
	und ist vollständig abgekoppelt vom \nameref{subsec:tls-handshake-protocol}.
	Zuerst werden die zu verschickenden Daten in handliche Blöcke aufgeteilt,
	welche optional auch komprimiert werden können.
	Jeder Nachricht wird dann ein Nachrichtenauthentifizierungscode (engl.\ \gls{mac}) zugewiesen,
	der zur Integritätsprüfung genutzt wird.
	Final werden noch genau diese Daten
	mit dem im \nameref{subsec:tls-handshake-protocol} ausgehandelten Schlüssel
	chiffriert.
	Sind die Daten nun vollständig verschlüsselt und verschickt,
	wird die empfangende Seite das Protokoll noch einmal rückwärts durchlaufen,
	um die Daten wieder darstellen zu können.