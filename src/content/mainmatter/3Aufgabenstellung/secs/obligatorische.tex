\section{Obligatorische Anforderungen}\label{sec:obligatorische-anforderungen}

	Der \gls{webservice} muss den IT-Richtlinien des
	\glslink{bsi}{Bundesamts für Sicherheit in der Informationstechnik (BSI)} gerecht werden
	und somit den Bestimmungen des IT-Grundschutz-Kompendiums\footcite[][]{holgerschildt2022} entsprechen.
	Es müssen über eine \gls{rest}-Schnittstelle Methoden für den Zugriff
	auf die dahinter liegende Datenbank bereitgestellt werden.
	Das Ganze soll schnell und zuverlässig passieren und mehrere Anfragen pro Sekunde verarbeiten können.
	Wie aus der Beschreibung eines \gls{rest}-\glspl{webservice} hervorgeht,
	muss die \gls{api} zudem selbst sprechend sein (Stichwort: \gls{hateoas}) und
	-- soweit möglich --
	alle \crudOperationen{} widerspiegeln können.
	Um den \gls{webservice} testen und die Resultate grafisch darzustellen,
	unter anderem dafür,
	dass die Einbindung in den Simulator gezeigt wird,
	muss auch noch eine Benutzeroberfläche existieren,
	die optimalerweise der eines Cockpits ähnelt.
	Des Weiteren müssen alle Programme in C\# geschrieben werden
	und selbstredend auf dem Zielsystem laufen.