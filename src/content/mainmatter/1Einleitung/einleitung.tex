\chapter{Einleitung}\label{ch:einleitung}

	Ein Flugsimulator besteht aus verschiedenen Simulationskomponenten,
	von denen viele mit speziell aufbereiteten Daten versorgt werden müssen.
	Hervorzuheben sind dabei beispielsweise Simulationen von Kommunikations- und Navigationsgeräten
	und \glslink{fms}{Flight Management Systemen (FMS)}.
	Diese Komponenten müssen mit passenden Daten versorgt werden.
	Die Logistik dafür übernimmt dabei das \gls{lfid}-System,
	welches Kommunikations- und Navigationsdaten in einer Datenbank verwaltet.
	Diese wird zentral beim \gls{ZSimNav} gepflegt
	und kann von angeschlossenen Simulatoren genutzt werden.
	Am Simulator selbst werden diese Daten noch in speziellen Formaten genutzt,
	welche allerdings nicht mehr mit der größeren Datenmenge zurecht kommen
	und deshalb obsolet sind.
	Die Idee ist es daher,
	einen Navigations- und Kommunikationsdatenservice im Simulator bereitzustellen,
	sodass unterschiedliche Simulatorkomponenten ihre benötigten Daten von dort beziehen.
	Dieser Service soll webbasiert sein,
	sich an der \gls{rest}-Architektur orientieren
	und den Ansprüchen der Bundeswehr in Bezug auf IT-Sicherheit gerecht werden.
	Hierbei spielt auch die \gls{authentifizierung} eine große Rolle
	sowie die nutzbaren Transportprotokolle.

	Das Kapitel \nameref{ch:grundlagen} stellt eine etwas allgemeiner
	gehaltene Beschreibung und Erklärung wichtiger Begriffe dar.
	So wird zuerst auf das Internetreferenzmodell eingegangen,
	woraufhin das \gls{http} und damit einhergehend danach die \gls{tls} folgt.
	Schließlich werden noch zwei Absicherungsverfahren vorgestellt.

	Die \nameref{ch:aufgabenstellung} behandelt obligatorische und optionale Anforderungen,
	welche anfangs an die Arbeit bzw.\ an das Produkt gestellt wurden.

	Es folgt die in vier Abschnitte unterteilte \nameref{ch:analyse} der Problemstellung.
	Die Datenhaltung behandelt grundsätzlich,
	wie und wo die Daten verfügbar gemacht werden.
	Das Zielsystem wird vor allem für die kompatiblen \glspl{framework} analysiert.
	Um die Sicherheit der Übertragung gewährleisten zu können,
	wird noch das IT Grundschutz Kompendium\footcite[Vgl.][]{holgerschildt2022} näher behandelt,
	woraufhin auch die Top 10 des \glslink{owasp}{Open Web Application Security Projects (OWASP)} folgen.

	Die \nameref{ch:realisierung} beinhaltet Erklärungen darüber,
	wie die \glspl{webservice} implementiert wurden
	und was zur Umsetzung der notwendigen Sicherheitsanforderungen verwendet wurde.

	In \nameref{ch:ergebnis-und-schlussfolgerungen} werden alle Resultate dieser Arbeit aufgezählt
	und Schlussfolgerungen über die weiteren Vorgehensweisen in Bezug darauf angerissen.

	Das schließende Kapitel \nameref{ch:zusammenfassung-und-ausblick} fasst alles Vorhergehende noch einmal zusammen
	und stellt einen Ausblick dar.

	Das Ziel dieser Arbeit wird es sein, einen \gls{rest}-basierten \gls{webservice} zu konzipieren,
	welcher sowohl den Anforderungen der Bundesregierung bezüglich IT-Sicherheit gerecht wird,
	als auch den Anforderungen des Zielsystems (Simulator),
	in dem dieser betrieben werden soll.
	Zudem werden die \gls{owasp} Top 10 angesprochen, analysiert und beachtet.
	Die Leser dieser Arbeit sollten danach in der Lage sein,
	eine Entscheidung zu treffen,
	welches Absicherungsverfahren bei unterschiedlichen Szenarien genutzt werden kann
	sowie (in Ansätzen) verstehen,
	wie ein \gls{webservice} gestaltet werden sollte und in \nameref{subsubsec:netcore21} implementiert werden könnte.
