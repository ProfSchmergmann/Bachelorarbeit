\subsubsection{Anlage: Kreuzreferenztabelle zu elementaren Gefährdungen}\label{subsubsec:anlage:app.3.1}

	Hier wird eine Kreuzreferenztabelle (s. \vref{tab:kreuzreferenztabelleApp31}) dargestellt,
	die zeigt welche elementaren
	Gefährdungen\footnote{\cite[Vgl. hierfür][Elementare Gefährdungen S. 1ff]{holgerschildt2022}}
	durch welche \hyperref[subsubsec:anforderungen-app.3.1]{Anforderungen}
	abgeschwächt oder gar eliminiert werden können.
	Folgende elementare Gefährdungen werden behandelt:
	\begin{compactitem}
		\item[\textbf{G 0.14}] Ausspähen von Informationen (Spionage)
		\item[\textbf{G 0.15}] Abhören
		\item[\textbf{G 0.18}] Fehlplanung oder fehlende Anpassung
		\item[\textbf{G 0.19}] Offenlegung schützenswerter Informationen
		\item[\textbf{G 0.21}] Manipulation von Hard- oder Software
		\item[\textbf{G 0.28}] Software-Schwachstellen oder -Fehler
		\item[\textbf{G 0.30}] Unberechtigte Nutzung oder Administration von Geräten und Systemen
		\item[\textbf{G 0.31}] Fehlerhafte Nutzung oder Administration von Geräten und Systemen
		\item[\textbf{G 0.43}] Einspielen von Nachrichten
		\item[\textbf{G 0.46}] Integritätsverlust schützenswerter Informationen
	\end{compactitem}
	Des Weiteren existiert eine Spalte \enquote{CIA},
	welche die Grundwerte der Informationssicherheit,
	wie auf Seite \pageref{itm:it-grundschutz-kompendium-verfügbarkeit} definiert,
	behandelt.
	C (Confidentiality) steht dabei für Vertraulichkeit,
	I (Integrity) für Korrektheit und
	A (Availability) für Verfügbarkeit.

