\subsubsection{Gefährdungslage}\label{subsubsec:gefaehrdungslage-app.3.1}
	Die Gefährdungslage dieses Bausteins ist noch einmal unterteilt.
	Im Folgenden werden relevante Gefährdungen angesprochen\footcite[Vgl.][APP.3.2 S. 1]{holgerschildt2022}.

	\myparagraph[APP.3.1.2.1]{Unzureichende Protokollierung von sicherheitsrelevanten Ereignissen}
	Die Protokollierung sicherheitsrelevanter Ereignisse ist erforderlich,
	um später die Ursachen eines bestimmten Ereignisses ermitteln
	und somit Schwachstellen, kritische Fehler,
	oder unerlaubte Änderungen nachvollziehen zu können.

	\myparagraph[APP.3.1.2.2]{Offenlegung sicherheitsrelevanter Informationen bei Webanwendungen und Webservices}
	Bei der Auslieferung von \glspl{webservice} ist darauf zu achten,
	keine sicherheitsrelevanten Daten offenzulegen,
	sprich Informationen über genutzte Programme,
	Systeme oder Versionen weiterzugeben.

	\myparagraph[APP.3.1.2.3]{Missbrauch einer Webanwendung durch automatisierte Nutzung}
	Ein \gls{webservice} kann durch automatisierte Anfragen missbraucht werden
	und somit eventuell Nutzerdaten preisgeben.
	Dadurch könnten Angreifer gültige Benutzernamen sammeln
	und sich dann damit durch mehrfaches Probieren
	von Passwörtern Zugriff zur Anwendung verschaffen.

	\myparagraph[APP.3.1.2.4]{Unzureichende Authentisierung}
	Meistens werden Rollenprofile angelegt,
	um Benutzern die Möglichkeit zu geben,
	bestimmte Ressourcen erreichen zu können,
	welche von anderen Profilen nicht erreicht werden können.
	Falls die \glslink{authentisierung}{Authentisierung} der Nutzer unzureichend ist,
	könnte ein Angreifer einfachen Zugriff erlangen
	und im schlimmsten Fall sogar auf die Nutzerdaten zugreifen,
	falls diese ebenfalls über diese Rolle erreichbar sind.