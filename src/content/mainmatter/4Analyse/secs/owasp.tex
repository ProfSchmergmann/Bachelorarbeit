\section{OWASP Top 10}\label{sec:owasp}
	Das \gls{owasp} ist eine öffentliche Gemeinschaft,
	welche in dreijährigen Abständen Top-10 Listen\footcite[Vgl.][]{owasp_top_ten} der
	am meisten ausgenutzten Sicherheitslücken veröffentlicht.
	Diese Listen sollen Unternehmen dazu animieren,
	sichere \webApplications{} bereitzustellen und sich gegen eben diese Bedrohungen zu schützen.
	Die zu diesem Zeitpunkt aktuellste Version ist die Liste aus dem Jahr 2021.
	Folgend werden die darin erhaltenen Sicherheitslücken aufgezählt,
	erläutert und erklärt, wie Entwickler Anwendungen dagegen schützen können.

	\myparagraph[A01:2021]{A01:2021 - Broken Access Control}
	Die sogenannten Fehler in der Zugriffskontrolle gewinnen gegenüber 2017 sehr an Bedeutung,
	da diese von Platz 5 auf Platz 1 gesetzt wurden.

	Falls Benutzer einer Anwendung mehr Rechte bekommen als eigentlich vorgesehen,
	führt dies unweigerlich zur Freigabe von Daten,
	für die diese nicht \glslink{autorisierung}{autorisiert} sind.
	Das könnten Angreifer über das Umgehen der Zugriffskontrollen erreichen,
	indem beispielsweise die \gls{url} verändert wird,
	die mitgeschickten Metadaten manipuliert oder das \gls{jwt} erneut verwendet oder modifiziert wird.

	Schon bei der Entwicklung müssen daher Rechte,
	Rollen und Regeln klar definiert sein,
	wie sich Benutzer \glslink{authentifizierung}{authentifizieren} können und
	welche Rechte an welche Benutzer vergeben werden dürfen.
	Dies sollte zudem durch Protokollierung der Ereignisse und
	Invalidierung der \glspl{jwt} sichergestellt werden.

	\newpage

	\myparagraph[A02:2021]{A02:2021 - Cryptographic Failures}
	Vorher als \enquote{Sensitive Data Exposure}
	oder \enquote{Verlust der Vertraulichkeit sensibler Daten} bekannt,
	rücken die \enquote{kryptographischen Fehler} auf Platz 2 der Liste.

	In der Regel werden hierbei \glspl{manInTheMiddleAngriff} ausgeführt,
	welche darauf abzielen, nicht die Verschlüsselung selbst zu brechen,
	sondern die Daten der Übertragung zur Kompromittierung zu nutzen.
	Des Weiteren kann durch das Verwenden unsicherer Protokolle,
	älterer Verschlüsselungsverfahren oder keiner verbindlich erzwungener Verschlüsselung,
	die Verwundbarkeit der Anwendung erhöht werden.

	Verhindern kann man diese Bedrohung durch das Vermeiden von unnötigem Speichern von sensiblen Daten,
	Deaktivieren des Caches für diese Daten,
	oder durch die Nutzung sicherer Transportprotokolle.

	\myparagraph[A03:2021]{A03:2021 - Injection}
	Diese Bedrohung ist von Platz 1 auf 3 herabgesetzt worden.

	Falls die \webApplication{} oder der \gls{webservice} eine Datenbankanbindung verwaltet,
	ist \enquote{Injection} eine beliebte Angriffsstrategie.
	Angreifer versuchen durch gezielte Ausnutzung von \gls{sql}-Queries Daten offenzulegen,
	zu kompromittieren oder zu löschen.
	Mit einer gelungenen Injection kann beispielsweise auch ein Administratorkonto vorgespielt werden,
	was natürlich fatal wäre.

	Durch korrekte Nutzung eines \gls{orm}-\glspl{framework} kann diese Bedrohung verringert werden.
	Zudem sollten Eingabedaten immer auf nicht erlaubte Buchstaben, sowie Zeichen und Länge geprüft werden.

	\myparagraph[A04:2021]{A04:2021 - Insecure Design}
	Die neue Kategorie \enquote{Insecure Design} bezieht sich auf Fehler in der Architektur und
	im Design von Anwendungen.

	Hierbei werden Sicherheitslücken externer \glspl{framework} und Design-Patterns ausgenutzt.
	Das \gls{owasp} versucht dadurch,
	Entwickler von Frameworks auf Sicherheitslücken aufmerksam zu machen und
	sichere Design-Patterns zu fördern.

	Verhindert werden kann das durch die Nutzung mehrfach getesteter Frameworks,
	sowie die Entwicklung sicherer Entwurfsmuster.

	\myparagraph[A05:2021]{A05:2021 - Security Misconfiguration}
	Diese Kategorie ist im Vergleich zum Jahr 2017 einen Platz weiter nach oben gerutscht.
	Sie beinhaltet Lücken in der Konfiguration von Firewalls, \glspl{webservice}
	und \webApplications.

	Angreifer könnten hierbei versuchen sich über Standardpasswörter Zugang zum System zu verschaffen,
	oder über Listings der Ordnerpfade die kompilierten Klassen finden,
	diese dann dekompilieren,
	um über \enquote{Reverse Engineering} eine Sicherheitslücke zu finden.

	Entwickler können eine derartige Sicherheitslücke minimieren bzw.\ eliminieren,
	indem sie unnötige Features, \glspl{framework} oder Komponenten deinstallieren
	und eine \enquote{abgespeckte} Anwendung ausliefern.
	Weiterhin können über regelmäßige Updates des genutzten Systems/\glspl{framework}
	veraltete Sicherheitslücken geschlossen werden.

	\myparagraph[A06:2021]{A06:2021 - Vulnerable and Outdated Components}
	Am zweitstärksten gewinnt diese Kategorie an Bedeutung,
	da ihre Wichtigkeit um 3 Stufen gestiegen ist.
	Erfasst wird die Nutzung von Komponenten mit bekannten Schwachstellen.

	Ein aktuelles Beispiel davon ist die am 10.\ Dezember 2021 bekanntgewordene Sicherheitslücke
	des Logging-\glspl{framework} \enquote{Log4j} von Java.
	Durch die interne Struktur der Sprache können so Exploits vorgenommen und der Host-Rechner kontrolliert werden.

	Unternehmen sollten daher immer darüber informiert sein,
	welche externen Bibliotheken genutzt werden,
	wie diese funktionieren und ob diese auf dem neuesten Stand sind.
	Im Falle eines Logging-\glslink{framework}{Frameworks} oder anderen Basiskomponenten
	ist es also manchmal auch von Vorteil,
	den Code entweder selbst zu schreiben oder Klassen (wie z.\ B.\ java.util.logging)
	direkt aus dem \gls{jdk} zu nutzen.

	\myparagraph[A07:2021]{A07:2021 - Identification and Authentication Failures}
	Sehr an Bedeutung verloren haben die \enquote{Fehler in der \gls{authentifizierung}}.

	Angriffe auf die \gls{authentifizierung} sind gut erforscht,
	wodurch es bereits sehr viele Skripte gibt,
	welche Wörterbücher als Datengrundlage nutzen,
	um durch \gls{bruteforce}-Angriffe Benutzernamen und Passwörter herauszufinden.
	Besonders gefährdet sind auch \glspl{webservice} mit nicht erlöschenden Session-Tokens,
	welche dann für die falsche \gls{authentifizierung} genutzt werden können.

	Sofern möglich, sollten daher folgende Themen implementiert sein,
	um der \gls{authentifizierung} einen höheren Sicherheitsstandard mitzugeben.
	Für die Risikominimierung automatisierter Angriffe
	sollte eine Mehrfaktor-\gls{authentisierung} implementiert werden
	und im Zuge dessen sollten auch alle Passwörter für ein neu angelegtes Nutzerkonto
	gegen eine Liste der 10000 beliebtesten Passwörter geprüft werden.
	Zusätzlich dazu dürfen im Auslieferungszustand keine Standardbenutzer
	und/oder administrative Nutzer vorhanden sein,
	wobei auch bei fehlgeschlagenen Anmeldeversuchen immer die gleiche Fehlermeldung zurückgegeben werden soll,
	um somit die Nutzernamen nicht erraten zu können.

	\myparagraph[A08:2021]{A08:2021 - Software and Data Integrity Failures}
	Die zweite neue Kategorie bezieht sich auf Sicherheitslücken in Software-Updates,
	kritischen Daten und \glspl{ciCdPipeline} ohne Integritätsprüfung.
	Hierzu zählt unter anderem die veraltete Kategorie aus 2017 \enquote{A08:2017 - Insecure Deserialization}.

	Beispielsweise kann im Prozess der Deserialisierung angesetzt werden,
	um mit fehlerhaften Routinen Schadcode auf dem \gls{webservice} ausführen zu können.
	Des Weiteren kann über eine fehlende Prüfung neuer Pakete,
	welche automatisiert nachgeladen werden,
	externer Code auf einem bestehendem System installiert und ausgeführt werden.

	Digitale Signaturen oder ähnliche Mechanismen für die Validierung der Pakete sind daher erforderlich,
	um solche Risiken auszuschließen.
	Zudem sollte -- gerade im Kontext von \webApplications{} -- sichergestellt werden,
	dass Paketmanager (wie z.\ B.\ Maven, Gradle oder npm) Code nur aus vertrauenswürdigen Quellen laden.

	\myparagraph[A09:2021]{A09:2021 - Security Logging and Monitoring Failures}
	Die damalige Kategorie \enquote{A10:2017 - Insufficient Logging \& Monitoring} beinhaltet vor allem
	unzureichende Protokollierung von Anmeldeversuchen oder falsche Sichtbarkeit von Logging-Informationen.

	Angreifer können also automatisiert und mit großer Anzahl an Tests versuchen,
	Anmeldedaten zu erraten.
	Weiterhin könnten Logs abgegriffen und interpretiert werden.

	Es ist also dringend erforderlich,
	Anmeldeversuche zu protokollieren und die richtigen Schlüsse daraus zu ziehen,
	beispielsweise die \gls{ip}-Adresse zu sperren oder ähnliches.
	Weiterhin muss darauf geachtet werden,
	wohin die Logs geschrieben werden,
	da diese Informationen für außenstehende Personen nicht zugänglich sein dürfen.

	\myparagraph[A10:2021]{A10:2021 - Server-Side Request Forgery (SSRF)}
	Die \gls{ssrf} ist die dritte neue Bedrohung in 2021
	und bezieht sich auf die Kontrolle des \glspl{webservice} über Requests,
	welche vom Angreifer so manipuliert werden,
	dass die Applikation anderen Code als geplant ausführt oder sogar geschützte Daten preisgibt.

	Ausgenutzt werden kann dies,
	indem zum Beispiel \gls{url}-Anforderungen durch \lstinline!127.0.0.1! oder \enquote{localhost} ersetzt werden
	und diese dadurch an das eigene Lookup-Interface weitergeleitet werden.

	Schutzmechanismen dagegen sind rar,
	da sich unter anderem nicht einmal Bibliotheken einig sind,
	wie gewisse \glspl{uri} zu interpretieren sind.
	Es könnte allerdings -- je nach Anwendung -- entweder der Whitelist,
	oder der Blacklist Ansatz für die Validierung der \glspl{url} verwendet werden.
