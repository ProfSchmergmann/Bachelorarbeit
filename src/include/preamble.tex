\usepackage{iftex}

\ifluatex
\usepackage{fontspec}
\else
\ifpdftex
\usepackage[utf8]{inputenc}
\usepackage[T1]{fontenc}
\usepackage{lmodern}
\pdfminorversion=7
\fi
\fi

\usepackage[ngerman]{babel}

% Für KOMA-Script
\usepackage{scrhack}

% Farben
\usepackage[x11names]{xcolor}
\definecolor{midblue}{rgb}{0,0,.7}
\definecolor{midred}{rgb}{.7,0,0}
\definecolor{midgreen}{rgb}{0,.7,0}

% Andere wichtige Pakete
\usepackage{pdfpages}
\usepackage{csquotes}
\usepackage{xltabular}
\usepackage{array}
\newcolumntype{Y}{>{\centering\arraybackslash}X}
\usepackage{diagbox}
\usepackage[labelfont=bf,textfont=sl, skip=15pt]{caption}
\usepackage{amsmath}
\usepackage{varioref}
\usepackage{geometry}

% Zeilenabstand zwischen Aufzählungen verringern
\usepackage{paralist}

% Bilder
\usepackage{graphicx}
\graphicspath{{include/img}}

% Tikz Graphiken
\usepackage{pgfplots}
\usetikzlibrary{shapes.multipart}
\usetikzlibrary{positioning}
\usetikzlibrary{shadows}
\usetikzlibrary{calc}
\tikzset{every picture/.style={line width=0.75pt},
	pics/entity/.style n args={3}{code={%
		\node[draw,
			rectangle split,
			rectangle split parts=2,
			text height=1.5ex,
		] (#1)
		{#2 \nodepart{second}
			\begin{tabular}{>{\raggedright\arraybackslash}p{8.5em}}
				#3
			\end{tabular}
		};%
	}},
	pics/entitynoatt/.style n args={2}{code={%
		\node[draw,
			text height=1.5ex,
		] (#1)
		{#2};%
	}}} %set default line width to 0.75pt
\usepackage{tikzpagenodes}
\pgfplotsset{compat = newest}
\usepackage{pgf-umlsd}
\usepgflibrary{arrows}

% Einrückung verhindern
\setlength{\parindent}{0em}

% Listings
\usepackage{listings}

\renewcommand{\lstlistlistingname}{Verzeichnis der Programmlistings}

\newcommand{\inlinelst}[1]{{\footnotesize\ttfamily#1}}

\lstset{
	backgroundcolor=\color{white},
	basicstyle=\ttfamily\footnotesize,
	breakatwhitespace=false,
	breaklines=true,
	captionpos=t,
	columns=fullflexible,
	frame=single,
	numbers=left,
	numbersep=5pt,
	numberstyle=\tiny\color{gray},
	rulecolor=\color{black},
	showspaces=false,
	showstringspaces=false,
	showtabs=false,
	stepnumber=1,
	tabsize=2,
	xleftmargin=.04\textwidth,
	xrightmargin=.02\textwidth
}

\lstdefinelanguage{XML}{
	morestring=[b]",
	moredelim=[s][\bfseries\color{Maroon}]{<}{\ },
	moredelim=[s][\bfseries\color{Maroon}]{</}{>},
	moredelim=[l][\bfseries\color{Maroon}]{/>},
	moredelim=[l][\bfseries\color{Maroon}]{>},
	morecomment=[s]{<?}{?>},
	morecomment=[s]{<!--}{-->},
	commentstyle=\color{green},
	stringstyle=\color{blue},
	identifierstyle=\color{red}
}

% JavaScript
\lstdefinelanguage{JavaScript}{
	keywords={typeof, new, true, false, catch, function, return, null, catch,
	switch, var, if, in, while, do, else, case, break},
	keywordstyle=\color{blue}\bfseries,
	ndkeywords={class, export, boolean, throw, implements, import, this},
	ndkeywordstyle=\color{darkgray}\bfseries,
	identifierstyle=\color{black},
	sensitive=false,
	comment=[l]{//},
	morecomment=[s]{/*}{*/},
	commentstyle=\color{purple}\ttfamily,
	stringstyle=\color{red}\ttfamily,
	morestring=[b]',
	morestring=[b]"
}

\lstdefinestyle{CSharp} {
	language=[Sharp]C,
	basicstyle=\ttfamily\scriptsize,
	commentstyle=\color{green},
	morecomment=[s][\color{green}]{/*+}{*/},
	morecomment=[s][\color{green}]{/*-}{*/},
	stringstyle=\color{red},
	literate={\$}{\textcolor{red}{\$}}{1},
	keywordstyle=\color{blue},
	morekeywords={partial, var, value, get, set, abstract, event, new, struct,
	as, explicit, null, base, extern, object, this, bool, false, operator,
	break, finally, out, true, byte, fixed, override, try, float, params,
	typeof, catch, for, private, uint, char, foreach, protected, ulong,
	checked, goto, public, unchecked, class, readonly, unsafe, const,
	implicit, ref, ushort, continue, in, using, decimal, int, sbyte, virtual,
	default, interface, sealed, volatile, delegate, internal, short, void,
	do, is, sizeof, while, double, lock, stackalloc, long, static,
	enum, namespace, string, nameof, await, async},
%-------------------------------------------
%keywords like if/else, switch/case ...
	emphstyle=\color{violet},
	emph={if, else, return, throw, switch, case, Assert},
%-------------------------------------------
%Collection of local and global variables
	emphstyle={[2]\color{teal}},
	emph={[2]someLocalVariable, someGlobalVariable, Client, _testOutputHelper,
	CONNECTION_STRING, _authorizedClientFixture, SQL_QUERY, _routes, lfidContext,
	Logger, Mapper, Configuration, Version_2_1},
%-------------------------------------------
%Collection of class types
	emphstyle={[3]\color{cyan}},
	emph={[3]SomeOwnClassType, AuthorizedClientFixture, CountriesSqlTest, ApiControllerBase,
	CountriesWebTestAuthorized, DateTime, HttpClient, ITestOutputHelper, ActionDescriptor,
	Theory, Repeat, KeyValuePair, HttpContent, HttpResponseMessage, Uri, Task, IHostingEnvironment,
	SqlConnection, CountriesWebTestUnauthorized, IClassFixture, List, IDisposable, MainWindow,
	CountryModel, Program, Startup, ApiController, Produces, Route, Authorize, IMapper, JsonConvert,
	ILogger, LfidContext, IReadOnlyList, Controller, ApiControllerBase, ActionResult,
	IActionDescriptorCollectionProvider, CountriesController, HttpGet, HttpHead, IApplicationBuilder,
	IEnumerable, DataNotFoundException, IWebHostBuilder, Config, ApiResource, Secret, JObject,
	SidStarApproachModel, AirportModel, SidStarApproachEnum, Waypoint, FormUrlEncodedContent},
%-------------------------------------------
%Collection of methods
	emphstyle={[4]\color{brown}},
	emph={[4]ToLower, GetCountryModels, Dispose, Should_return_all_countries_from_Database,
	RequestTokenToAuthorizationServer, Close, WriteLine, DeserializeObject, Parse, Main,
	Should_return_all_countries_with_Authentication, ReadAsStringAsync, SendAsync, Build,
	Should_return_all_countries, NotEmpty, GetAsync, ExecuteReader, Open, Read, ToString,
	Run, CreateWebHostBuilder, CreateDefaultBuilder, ConfigureLogging, UseStartup, UseUrls,
	GetShadowCountries, ToListAsync, OrderBy, ThenBy, GroupBy, Select, FirstOrDefault, Map,
	AddDebug, AddConsole, GetApiResources, GetClients, Configure, ConfigureServices,
	AddIdentityServer, AddDeveloperSigningCredential, AddInMemoryApiResources, AddInMemoryClients,
	AddMvc, SetCompatibilityVersion, isDevelopment, UseDeveloperExceptionPage, useHsts,
	UseIdentityServer, UseHttpsRedirection, UseMvc, MapRoute, RequestTokenToAuthorizationServer,
	isSuccessStatusCode, GetWaypointsAsync, GetSortedLegsAsync, GetSidsStarsApproachesAsync,
	GetAirportsAsync, GetCountriesAsync, Content, Result},
}

% Kopf- und Fusszeile
\usepackage[
	headsepline,
	footsepline,
	autooneside=false,
	automark
]{scrlayer-scrpage}

% Zeilenabstand
\usepackage[onehalfspacing]{setspace}

% Abstand zwischen Kopfzeile und Kapitelüberschrift
\renewcommand*{\chapterheadstartvskip}{\vspace*{-0.75\baselineskip}}

% Um die Platzhalter zu leeren
\clearpairofpagestyles

% head definieren
\automark[section]{chapter}

\ihead{\leftmark} % Kopfzeile innen
\chead{} % Kopfzeile mitte
\ohead{\Ifstr{\leftmark}{\rightbotmark}{}{\rightbotmark}} % Kopfzeile aussen
\ifoot{\myAuthor, CAE GmbH Stolberg} % Fußzeile innen
\cfoot{} % Fußzeile mitte
\ofoot{\thepage} % Fußzeile aussen mit Seitenzahl

% Kapitelnummerierung in der Kopfzeile aus
\renewcommand*{\sectionmarkformat}{}

% Paragrafüberschriften
\newcommand{\myparagraph}[2][XXX]{\paragraph[#1]{#2}\label{par:#1}\mbox{}\\}
\newcommand{\mysubparagraph}[2][XXX]{\subparagraph[#1]{#2}\label{subpar:#1}\mbox{}\\}

% BibLaTeX
\usepackage[
	backend=biber,
	maxbibnames=99,
	style=verbose-ibid
]{biblatex}
\addbibresource{include/Bachelorarbeit.bib}

% Zeilenabstand im Literatureintrag zurücksetzen
\AtBeginBibliography{\singlespacing}
% Zeilenabstand zwischen den einzelnen Einträgen im Literaturverzeichnis setzen
\setlength{\bibitemsep}{1.5\itemsep}

\usepackage[
	pdftitle={\myTitle},
	pdfauthor={\myAuthor},
	pdfsubject={\myThesistype},
	colorlinks=true,
	linkcolor=midblue,
	citecolor=midgreen,
	urlcolor=midred
]{hyperref}
\usepackage[noabbrev]{cleveref}

\usepackage{bookmark}

% Glossary
\usepackage[acronym,toc,nopostdot]{glossaries}
\makenoidxglossaries % use TeX to sort

% #1 - reference e.g. api
% #2 - Short e.g. API
% #3 - Full name e.g. Application Programming Interface
% #4 - Description
\newcommand{\newdefineabbreviation}[4]
{
% Glossary entry
	\newglossaryentry{#1_glossary}
	{text={#2},
		long={#3},
		name={\glsentrylong{#1_glossary} (\glsentrytext{#1_glossary})},
		description={#4}
	}

% Acronym
	\newglossaryentry{#1}
	{type=\acronymtype,
		name={\glsentrytext{#1_glossary}}, % Short
		description={\glsentrylong{#1_glossary}}, % Full name
		first={\glsentryname{#1_glossary}\glsadd{#1_glossary}},
		see=[Glossar:]{#1_glossary} % Reference to corresponding glossary entry
	}
}
\loadglsentries{include/glossary}