% Akronyme ohne Erklärung

\newacronym{bsi}{BSI}{Bundesamt für Sicherheit in der Informationstechnik}
\newacronym{crud}{CRUD}{Create Read Update Delete}
\newacronym{ftp}{FTP}{File Transfer Protocol}
\newacronym{hateoas}{HATEOAS}{Hypermedia as the Engine of Application State}
\newacronym{http}{HTTP}{Hypertext Transfer Protocol}
\newacronym{https}{HTTPS}{Hypertext Transfer Protocol Secure}
\newacronym{icmp}{ICMP}{Internet Control Message Protocol}
\newacronym{ip}{IP}{Internet Protocol}
\newacronym{lfid}{LFID}{Luftfahrt Informations Datenbank}
\newacronym{owasp}{OWASP}{Open Web Application Security Project}
\newacronym{rest}{REST}{Representational State Transfer}
\newacronym{ssl}{SSL}{Secure Sockets Layer}
\newacronym{tcp}{TCP}{Transmission Control Protocol}
\newacronym{tls}{TLS}{Transport Layer Security}
\newacronym{udp}{UDP}{User Datagram Protocol}
\newacronym{smtp}{SMTP}{Simple Mail Transfer Protocol}
\newacronym{ssrf}{SSRF}{Server-Side-Request-Forgery}
\newacronym{ZSimNav}{ZSimNav}{Zentrum für Simulations- und Navigationsunterstützung Fliegende Waffensysteme der Bundeswehr}
\newacronym{iso}{ISO}{International Standards Organisation}
\newacronym{osi}{OSI}{Open Systems Interconnection}
\newacronym{gui}{GUI}{Graphical User Interface}

% Akronyme mit Erklärung

\newdefineabbreviation
{api}{API}{Application Programming Interface}
{Satz von Befehlen, Protokollen und Funktionen,
	welche Entwickler für die Erstellung von Software zur Ausführung
	allgemeiner Operationen verwenden können.
	Diese Programmierschnittstelle stellt somit die Grundlage
	für die Kommunikation zwischen Anwendungen bereit.}

\newdefineabbreviation
{html}{HTML}{Hypertext Markup Language}
{Sprache zur Darstellung von Inhalten über einen Web Browser}

\newdefineabbreviation
{uri}{URI}{Uniform Resource Identifier}
{Eindeutige Adressierung von abstrakten und
physikalischen Ressourcen im Internet\\
\glsname{uri}: \glsname{url}s $\cup$ \glsname{urn}s}

\newdefineabbreviation
{url}{URL}{Uniform Resource Locator}
{Adressierung von Informationsobjekten mit Festlegung
des Zugangs-Protokolls (Ort der Ressource)}

\newdefineabbreviation
{urn}{URN}{Uniform Resource Name}
{Adressierung von Objekten ohne ein Protokoll festzulegen
	(Eindeutige und gleichbleibende Referenz - Name der Ressource)}

\newdefineabbreviation
{ietf}{IETF}{Internet Engineering Task Force}
{Organisation zur Verbesserung und Weiterentwicklung der Funktionsweise des Internets}

\newdefineabbreviation
{jwt}{JWT}{JSON Web Token}
{Token, welches Behauptungen (claims) über den Benutzer enthält.
Es stellt sich in einer codierten Form dar und kann durch Applikationen decodiert und validiert werden}

\newdefineabbreviation
{orm}{ORM}{Object Relational Mapping}
{Technik, um Objekte aus objektorientierten Programmiersprachen in relationalen Datenbanken abzulegen}

\newdefineabbreviation
{jdk}{JDK}{Java Development Kit}
{Zusammensetzung aus Java-Compiler, Java-Debugger und einigen anderen Entwicklungswerkzeuge,
	welche die Entwicklung von Java Applikationen ermöglichen}

\newdefineabbreviation
{ieee}{IEEE}{Institute of Electrical and Electronic Engineers}
{Weltweiter Berufsverband von Ingenieuren, Wissenschaftlern und Technikern,
	welcher vor allem für Standardisierungen und andere wissenschaftliche Veröffentlichungen bekannt ist}

\newdefineabbreviation
{json}{JSON}{JavaScript Object Notation}
{Maschinenlesbare Sprache zur Darstellung von Objekten.
Bei JavaScript kann dies unter anderem für eine
Instanziierung neuer Objekte genutzt werden.}

\newdefineabbreviation
{linq}{LINQ}{Language Integrated Query}
{Sprachkonstrukt in C\#, bei dem die Abfragen \glsname{sql}-Befehlen ähneln}

\newdefineabbreviation
{sql}{SQL}{Structured Query Language}
{Sprache zur Definition der Struktur einer Datenbank,
	sowie Abfragen der enthaltenen Daten}

\newdefineabbreviation
{star}{STAR}{Standard Terminal Arrival Routes}
{Festgelegte Strecken,
	welche ein Flugzeug zum Erreichen eines Ziels abfliegen kann}

\newdefineabbreviation
{sid}{SID}{Standard Instrument Departure}
{Festgelegte Strecken,
	welche ein Flugzeug nach dem Start abfliegen kann,
	um eine Luftfahrtstrecke erreichen zu können}

\newdefineabbreviation
{waf}{WAF}{Web Application Firewall}
{Verfahren um \glspl{webservice} vor Angriffen über das \glsname{http} zu schützen.
Hierbei wird eine zusätzliche Ebene geschaffen,
	welche alle Anfragen auf Schadcode untersucht.}

\newdefineabbreviation
{iis}{IIS}{Internet Information Services}
{Diensteplattform von Microsoft für die Bereitstellung von Daten,
	Dokumenten und Funktionen über Internetprotokolle}

\newdefineabbreviation
{wpf}{WPF}{Windows Presentation Foundation}
{\gls{framework} zur einfachen Erstellung einer Benutzeroberfläche,
	welches in .Net Core seit Version 3 zur Verfügung steht
	und Bestandteil des .NET Frameworks ist.}

\newdefineabbreviation
{fms}{FMS}{Flight Management System}
{elektronisches Hilfsmittel für Piloten zur Steuerung und Navigation}

\newdefineabbreviation
{mac}{MAC}{Message Authentication Code}
{Code zur Integritätsprüfung einer Nachricht}

\newdefineabbreviation
{ssms}{SSMS}{SQL Server Management Studio}
{Programm von Microsoft zur Verwaltung einer \glsname{sql}-Datenbank}

\newdefineabbreviation
{dll}{DLL}{Dynamic Link Library}
{Dynamische Programmbibliothek, welche vor allem unter Windows ausführbar ist}

\newdefineabbreviation
{arinc}{ARINC}{Aeronautical Radio Incorporated}
{Eine -- vor allem in der Luftfahrt -- bekannte Firma,
	welche Standards für Protokolle und Datenformate veröffentlicht}

\newdefineabbreviation
{icao}{ICAO}{International Civil Aviation Organization}
{Zu deutsch \enquote{Internationale Zivilluftfahrtorganisation} ist eine Organisation,
	welche das Ziel hat, die internationale Luftfahrt zu fördern}

% Glossareinträge ohne Akronym

\newglossaryentry{isoOsiModell}{
	name=ISO/OSI-Referenzmodell,
	description={Modell für die Netzwerkprotokolle als Darstellung
	über eine Schichtenarchitektur,
	siehe auch~\vref{fig:schichtenmodell}}
}

\newglossaryentry{framework}{
	name=Framework,
	description={Zu deutsch \enquote{Rahmenwerk},
	was eine wiederverwendbare Struktur bereitstellt,
	welche für die Entwicklung verschiedener Programme genutzt werden kann}
}

\newglossaryentry{authentisierung}{
	name=Authentisierung,
	description={Ein Nutzer legt Nachweise vor,
	welche dessen Identität bestätigen sollen
	(Behauptung einer Identität)}
}

\newglossaryentry{authentifizierung}{
	name=Authentifizierung,
	description={Stellt die Prüfung der behaupteten \glsname{authentisierung} dar
	(Verifizierung der Identität)}
}

\newglossaryentry{autorisierung}{
	name=Autorisierung,
	description={Nach erfolgreicher \glsname{authentifizierung} werden spezielle Rechte
	an den Nutzer vergeben
	(Vergabe oder Verweigerung von Rechten)}
}

\newglossaryentry{webservice}{
	name={Web Service},
	description={Softwaresystem,
	welches die Maschine-zu-Maschine Kommunikation über ein Netzwerk realisiert\footcite[Vgl.][]{WebServiceW3C}.}
}

\newglossaryentry{bruteforce}{
	name={Brute-Force},
	description={Zu deutsch \enquote{rohe Gewalt} was im Zusammenhang mit der Informatik häufig genutzt wird,
	um eine Problemlösung zu beschreiben,
	bei der keine effizienten Algorithmen bekannt sind,
	sondern einfach alle möglichen Lösungen durchprobiert werden}
}

\newglossaryentry{ciCdPipeline}{
	name={CI/CD-Pipeline},
	description={Steht für \enquote{Contigeous Integration/Contigeous Delivery}-Pipeline,
	welche mehrere Schritte umfasst,
	die für die Bereitstellung einer neuen Softwareversion ausgeführt werden müssen}
}

\newglossaryentry{cipher-suite}{
	name={Cipher Suite},
	description={Sammlung kryptografischer Algorithmen,
	welche für die Verschlüsselung von Nachrichten verwendet werden}
}

\newglossaryentry{ssl-zertifikat}{
	name={SSL-Zertifikat},
	plural={SSL-Zertifikate},
	description={Datensatz, welcher von einer Zertifizierungsstelle an
	Unternehmen und / oder Betreiber von Webseiten ausgestellt wird.
	Dieser enthält zahlreiche Daten,
	unter anderem den Namen des Ausstellers, die Gültigkeitsdauer,
	eine Seriennummer oder auch den Fingerabdruck der Verschlüsselung.}
}

\newglossaryentry{cookie}{
	name={Cookie},
	description={Kleine Textdateien, welche Nutzerdaten beinhalten, die vom Browser gespeichert werden}
}

\newglossaryentry{manInTheMiddleAngriff}{
	name={Man-in-the-Middle Angriff},
	plural={Man-in-the-Middle Angriffe},
	description={Angriffsszenario, bei dem ein Dritter ein System zwischen zwei Kommunikationspartnern platziert,
	um sensible Daten abfangen zu können}
}

\newglossaryentry{constructorInjection}{
	name={Konstruktor Injektion},
	description={Wird als Methodik bezeichnet,
	die in \glspl{framework} genutzt wird um Klassen oder Attribute zu kreieren,
	welche wiederum von der aufrufenden Klasse gebraucht werden}
}

\newglossaryentry{singleton}{
	name={Singleton},
	description={Entwurfsmuster, wobei nur ein einziges Objekt einer Klasse existiert,
	welches als Singleton bezeichnet wird.}
}

\newglossaryentry{openApiSpec}{
	name={OpenAPI Specification},
	description={Offene Spezifikation einer \glsname{api},
	welche genutzt werden kann um Client Code in verschiedenen Sprachen zu generieren}
}

\newglossaryentry{saltedHashVerfahren}{
	name={Salted-Hash-Verfahren},
	description={Wird als zusätzliche Absicherung z.\ B.\ bei der Speicherung von Passwörtern genutzt.
	Bei diesem Verfahren wird dem Klartext
	vor der Weiterverarbeitung eine zufällig generierte Zeichenfolge hinzugefügt. }
}